\documentclass{article}

\usepackage{amssymb} %to use beautiful mathematical symbols like the set of real numbers $\mathbb{R}$

\usepackage[utf8]{inputenc}
\usepackage[T1]{fontenc}

\usepackage[english,russian]{babel}


\begin{document}
    \section{Задание 3C}
    \subsection{Задача}
    Извлечь из выбранного текста неразрывные двухсловные коллокации, которые подчиняются заданным грамматичеким правилам и содержат заданное слово.
    \subsection{Выбор данных и инстументов}
    Для анализа был выбран роман Л.Н. Толстого "Война и Мир", все тома. Файл WarAndPeace.txt с текстом произведения включен в репозиторий. Для предварительной обработки текста был использован токенизатор xtok (Елена Игоревна в начале занятий дала ссылку на него: https://github.com/alesapin/xtok) и морфоанализатор pymorphy. Грамматические образцы будут перечислены позже, а искомое слово было "жизнь".
    \subsection{Работа программы}

    Обработка текста начинается с разбиения его на токены. Однако весь текст слишком большой, и токенизатор xtok выдает ошибку переполнения памяти при попытке подать в него текст целиком. Чтобы решить проблему, я использовал предварительный этап - разбил текст по пробелу. Далее знаки препинания и слова с дефисами, а также все, что угодно, что может быть разделено пробелами, передавалось для обработки в xtok.tokenize. Функция isplit возвращает генератор, который пробегает по всем кусочкам текста между пробелами. Она была взята со stackoverflow (https://stackoverflow.com/questions/3862010/is-there-a-generator-version-of-string-split-in-python), хотя могла бы быть написана и мной, просто подвернулось готовое решение.

    На каждой итерации очередной элемент из isplit передается в токенизатор. Токены с типом b'WORD' далее анализируются токенизатором, а все остальное игнорируется. Даже более того, любой токен типа не b'WORD' прерывает словосочетание. Например, в предложении "Радует меня жизнь: моя судьба в моих руках." не будет найдено словосочетание "жизнь моя", тк слова разделены двоеточием.

    В программе был задан глобальный словарь, который сохранял количество вхождений каждой словоформы в текст (после токенизации каждый токен b'WORD' добавлял единичку к соответствующему разделу словаря, см registerWord). Использвовались именно словоформы, а не леммы, то есть "жизнь" и "жизнью" считались по отдельности.

    Результат анализа предыдущего токена сохранялся во временной переменной, и если оба токена являлись словом, то производится анализ на соответствие словосочетания грамматическому образцу. Грамматический образец записывался следующим образом:\\
    \(\{'\_POS':'ADJF', 'case': True, 'gender': True, 'number': True\}, \{'\_POS':'NOUN'\}\)

    Он состоит из кортежа из двух элементов. Первый элемент кортежа описывает первое слово, второй - второе. Слово описывается словарем, где ключами являются названия морфологических тегов (согласно системе тегов opencorpora), а значением - искомые значения тегов. Если в первом слове вместо значения стоит "True", то требуется, чтобы второе слово имело такое же значение выбранного тега. Таким образом реализуется согласование. В примере правило описывает словосочетание прилагательное + существительное, которые согласуются по роду, числу и падежу.

    Полный список образцов:
    \begin{itemize}
        \item Из примера: прилагательное + существительное, которые согласуются по роду, числу и падежу
        \item Деепричастие + существительное
        \item Существительное в именительном падеже + глагол с солгасованием по числу и роду
        \item Существительное в именительном падеже + существительное в родительном падеже
    \end{itemize}

    Если словосочетание удовлетворяло хотя бы одному из грамматических образцов (см testOnRules), то оно сохранялось или увеличивало соответсвующий счетчик на 1 (см registerPair).

    После пробега по всему тексту, получалось два словаря: абсолютные частотности всех словоформ и абсолютные частотности всех удоавлетворяющих шаблонам словосочетаний. На основе этих данных считатись метрики $MI$, $MI_3$, $t-score$, $loglog-likelohood$, $dice$. Выражения для метрик достаточно красиво смотрятся в коде. Строчки 206-219 содержат функции, которые использовались для расчета метрик. Далее по каждой метрике выявлялся топ-20 словосочетаний относительно выбранной метрики.

    \subsection{Результат работы}

    При обработке Войны и Мира было выявлено 194864 словоформы. Это параметр $N$ для расчета метрик.

    \begin{center}
    \begin{tabular}{|c|c|c|c|c|}
    \hline \textbf{словосочетание} & \textbf{$MI$} & \textbf{\#словосочетаний} & \textbf{\#словоформ 1} & \textbf{\#словоформ 2} \\ \hline

ежедневною жизнью & 13.66 & 1  & 1  & 15 \\ \hline
целою жизнью & 13.66 & 1  & 1  & 15 \\ \hline
настоящею жизнью & 12.66 & 1  & 2  & 15 \\ \hline
беспутную жизнь & 11.017 & 1  & 1  & 94 \\ \hline
походная жизнь & 11.017 & 1  & 1  & 94 \\ \hline
спасая жизнь & 11.017 & 1  & 1  & 94 \\ \hline
Тихая жизнь & 11.017 & 1  & 1  & 94 \\ \hline
домашняя жизнь & 11.017 & 1  & 1  & 94 \\ \hline
распутную жизнь & 11.017 & 1  & 1  & 94 \\ \hline
развратную жизнь & 11.017 & 1  & 1  & 94 \\ \hline
жизнь наблюдала & 11.017 & 1  & 94  & 1 \\ \hline
жизнь велась & 11.017 & 1  & 94  & 1 \\ \hline
скотскую жизнь & 11.017 & 1  & 1  & 94 \\ \hline
тихая жизнь & 11.017 & 1  & 1  & 94 \\ \hline
настоящая жизнь & 11.017 & 1  & 1  & 94 \\ \hline
петербургскую жизнь & 11.017 & 1  & 1  & 94 \\ \hline
Семейная жизнь & 11.017 & 1  & 1  & 94 \\ \hline
мгновенная жизнь & 11.017 & 1  & 1  & 94 \\ \hline
отрадненская жизнь & 11.017 & 1  & 1  & 94 \\ \hline
привычная жизнь & 11.017 & 1  & 1  & 94 \\ \hline


    \end{tabular}
    \end{center}

    \begin{center}
    \begin{tabular}{|c|c|c|c|c|}
    \hline \textbf{словосочетание} & \textbf{$MI_3$} & \textbf{\#словосочетаний} & \textbf{\#словоформ 1} & \textbf{\#словоформ 2} \\ \hline

свою жизнь & 15.29 & 16  & 211  & 94 \\ \hline
ежедневною жизнью & 13.66 & 1  & 1  & 15 \\ \hline
целою жизнью & 13.66 & 1  & 1  & 15 \\ \hline
новой жизни & 13.19 & 5  & 20  & 130 \\ \hline
будущую жизнь & 13.18 & 3  & 6  & 94 \\ \hline
настоящею жизнью & 12.66 & 1  & 2  & 15 \\ \hline
образ жизни & 12.22 & 4  & 20  & 130 \\ \hline
прежнюю жизнь & 12.017 & 2  & 4  & 94 \\ \hline
внутренняя жизнь & 12.017 & 2  & 4  & 94 \\ \hline
счастие жизни & 11.90 & 4  & 25  & 130 \\ \hline
будущая жизнь & 11.43 & 2  & 6  & 94 \\ \hline
новую жизнь & 11.21 & 2  & 7  & 94 \\ \hline
беспутную жизнь & 11.017 & 1  & 1  & 94 \\ \hline
походная жизнь & 11.017 & 1  & 1  & 94 \\ \hline
спасая жизнь & 11.017 & 1  & 1  & 94 \\ \hline
Тихая жизнь & 11.017 & 1  & 1  & 94 \\ \hline
домашняя жизнь & 11.017 & 1  & 1  & 94 \\ \hline
распутную жизнь & 11.017 & 1  & 1  & 94 \\ \hline
развратную жизнь & 11.017 & 1  & 1  & 94 \\ \hline
жизнь наблюдала & 11.017 & 1  & 94  & 1 \\ \hline


    \end{tabular}
    \end{center}

    \begin{center}
    \begin{tabular}{|c|c|c|c|c|}
    \hline \textbf{словосочетание} & \textbf{$t$-score} & \textbf{\#словосочетаний} & \textbf{\#словоформ 1} & \textbf{\#словоформ 2} \\ \hline

свою жизнь & 3.97 & 16  & 211  & 94 \\ \hline
новой жизни & 2.23 & 5  & 20  & 130 \\ \hline
образ жизни & 1.99 & 4  & 20  & 130 \\ \hline
счастие жизни & 1.99 & 4  & 25  & 130 \\ \hline
всю жизнь & 1.98 & 4  & 81  & 94 \\ \hline
эту жизнь & 1.96 & 4  & 151  & 94 \\ \hline
его жизни & 1.93 & 6  & 1885  & 130 \\ \hline
будущую жизнь & 1.73 & 3  & 6  & 94 \\ \hline
вся жизнь & 1.71 & 3  & 58  & 94 \\ \hline
эта жизнь & 1.70 & 3  & 95  & 94 \\ \hline
своей жизни & 1.64 & 3  & 236  & 130 \\ \hline
прежнюю жизнь & 1.41 & 2  & 4  & 94 \\ \hline
внутренняя жизнь & 1.41 & 2  & 4  & 94 \\ \hline
будущая жизнь & 1.41 & 2  & 6  & 94 \\ \hline
новую жизнь & 1.41 & 2  & 7  & 94 \\ \hline
будущей жизни & 1.40 & 2  & 9  & 130 \\ \hline
истины жизни & 1.40 & 2  & 9  & 130 \\ \hline
ход жизни & 1.40 & 2  & 12  & 130 \\ \hline
пути жизни & 1.40 & 2  & 14  & 130 \\ \hline
своей жизнью & 1.40 & 2  & 236  & 15 \\ \hline


    \end{tabular}
    \end{center}

    \begin{center}
    \begin{tabular}{|c|c|c|c|c|}
    \hline \textbf{словосочетание} & \textbf{loglog-likelihood} & \textbf{\#словосочетаний} & \textbf{\#словоформ 1} & \textbf{\#словоформ 2} \\ \hline

свою жизнь & 116.74 & 16  & 211  & 94 \\ \hline
новой жизни & 42.74 & 5  & 20  & 130 \\ \hline
образ жизни & 32.91 & 4  & 20  & 130 \\ \hline
счастие жизни & 31.62 & 4  & 25  & 130 \\ \hline
будущую жизнь & 30.052 & 3  & 6  & 94 \\ \hline
всю жизнь & 26.71 & 4  & 81  & 94 \\ \hline
эту жизнь & 23.11 & 4  & 151  & 94 \\ \hline
вся жизнь & 20.23 & 3  & 58  & 94 \\ \hline
прежнюю жизнь & 20.035 & 2  & 4  & 94 \\ \hline
внутренняя жизнь & 20.035 & 2  & 4  & 94 \\ \hline
будущая жизнь & 18.86 & 2  & 6  & 94 \\ \hline
новую жизнь & 18.42 & 2  & 7  & 94 \\ \hline
эта жизнь & 18.097 & 3  & 95  & 94 \\ \hline
будущей жизни & 16.75 & 2  & 9  & 130 \\ \hline
истины жизни & 16.75 & 2  & 9  & 130 \\ \hline
ход жизни & 15.92 & 2  & 12  & 130 \\ \hline
пути жизни & 15.48 & 2  & 14  & 130 \\ \hline
ежедневною жизнью & 13.66 & 1  & 1  & 15 \\ \hline
целою жизнью & 13.66 & 1  & 1  & 15 \\ \hline
своей жизнью & 13.56 & 2  & 236  & 15 \\ \hline


    \end{tabular}
    \end{center}


    \begin{center}
    \begin{tabular}{|c|c|c|c|c|}
    \hline \textbf{словосочетание} & \textbf{dice} & \textbf{\#словосочетаний} & \textbf{\#словоформ 1} & \textbf{\#словоформ 2} \\ \hline

ежедневною жизнью & 0.12 & 1  & 1  & 15 \\ \hline
целою жизнью & 0.12 & 1  & 1  & 15 \\ \hline
настоящею жизнью & 0.11 & 1  & 2  & 15 \\ \hline
свою жизнь & 0.10 & 16  & 211  & 94 \\ \hline
прежней жизнью & 0.076 & 1  & 11  & 15 \\ \hline
счастливой жизнью & 0.071 & 1  & 13  & 15 \\ \hline
новой жизни & 0.066 & 5  & 20  & 130 \\ \hline
будущую жизнь & 0.06 & 3  & 6  & 94 \\ \hline
образ жизни & 0.053 & 4  & 20  & 130 \\ \hline
счастие жизни & 0.051 & 4  & 25  & 130 \\ \hline
моей жизнью & 0.046 & 1  & 28  & 15 \\ \hline
всю жизнь & 0.045 & 4  & 81  & 94 \\ \hline
прежнюю жизнь & 0.040 & 2  & 4  & 94 \\ \hline
внутренняя жизнь & 0.040 & 2  & 4  & 94 \\ \hline
Жизнь старого & 0.04 & 1  & 5  & 45 \\ \hline
будущая жизнь & 0.04 & 2  & 6  & 94 \\ \hline
новую жизнь & 0.039 & 2  & 7  & 94 \\ \hline
вся жизнь & 0.039 & 3  & 58  & 94 \\ \hline
своею жизнью & 0.035 & 1  & 41  & 15 \\ \hline
эту жизнь & 0.032 & 4  & 151  & 94 \\ \hline

    \end{tabular}
    \end{center}



    \subsection{Выводы}

    Была написана программа для поиска двухсловных неразрывных согласованных словосчетаний, которые удовлетворяют грамматическим образцам и максимизируют одну из метрик: $MI$, $MI_3$, $t-score$, $loglog-likelohood$.

    Можно заметить, что в топе относительно метрики $MI$ каждое словосочтание имеет хотя бы одно слово, которое встретилось в тексте всего 1 раз (кроме "настоящею жизнью"). То есть $MI$ поднимает словосочетания, которые содержат редкие слова.

    Модификация метрики $MI$ - $MI_3$ частично сглаживает описанный выше эффект, тк увеличивает значимость словосочетаний, которые встречаются более одного раза. В топе появились словосочетания, которые можно назвать коллокациями: свою жизнь, новой жизни, будущую жизнь, образ жизни, прежнюю жзнь, счастие жизни.

    В топе относительно метрики $t-score$, наоборот, нет ни одного словосочетания с уникальной словоформой. Все перечисленные ранее коллокации в топе по $t-score$ тоже присутствуют. Морфоанализатор pymorphy разбирает слово "эту" как прилагательное, слово частотное, поэтому в топе присутствуют разные варианты типа: эту жизнь, его жизни, вся жизнь, эта жизнь. Метрика $t-score$ поднимает в топ словосочетания, в которых присутствуют частотные слова.

    Топ относительно $loglog-likelihood$ совпадает с топом относительно $t-score$. Возможны некоторые перемещения, но в целом очень похожий результат.

    Метрика $dice$ дает компромисс между низкочастотными и высокочастотными словами. В топе есть и словосочетания со служебными словами: свою жизнь, эту жизнь. А также с низкочастотными словами: ежедневною жизнью, прежней жизнью.

    Мое субъективное мнение: метрика $dice$ в данном случае сработала лучше, чем остальные.

\end{document}
